\section{Radiant heat fluxes}
\label{shortwave}

As was seen in \S\ref{Growth}, the model thermodynamics requires
fluxes of latent and sensible heat and longwave and shortwave
radiation.  We follow the lead of
\citet{Parkinson} in computing these terms.

\subsection{Shortwave radiation}

The Zillman equation for radiation under cloudless skies is:
\begin{equation}
   Q_o = {S \cos^2 Z \over (\cos Z + 2.7) e \times 10^{-5} + 1.085
   \cos Z + 0.10}
\end{equation}
where the variables are as in Table \ref{radvars}.  The cosine of the
zenith angle is computed using the formula:
\begin{equation}
   \cos Z = \sin \phi \sin \delta + \cos \phi \cos \delta \cos H\!A .
\end{equation}
The declination is 
\begin{equation}
   \delta = 23.44^{\circ} \times \cos \left[ (172 - {\rm day \, of \, year})
   \times 2 \pi / 365 \right]
\end{equation}
and the hour angle is
\begin{equation}
   H\!A = (12 \, {\rm hours - solar \, time}) \times \pi / 12 .
\end{equation}
The correction for cloudiness is given by
\begin{equation}
   SW\!\!\downarrow = Q_o ( 1 - 0.6 c^3) .
\end{equation}
%An estimate of the cloud fraction $c$ will be provided by Jennifer
%Francis \citep{Francis00}.

\begin{table}[t]
\hspace{11 mm}
\vbox{
\begin{tabular}{|c|c|l|} \hline
Variable & Value & Description \\ \hline

$(a,b)$ & (9.5, 7.66) & vapor pressure constants over ice \\
$(a,b)$ & (7.5, 35.86) & vapor pressure constants over water \\
$c$ && cloud cover fraction \\
$C_E$ & $1.75 \times 10^{-3}$ & transfer coefficient for latent heat \\
$C_H$ & $1.75 \times 10^{-3}$ & transfer coefficient for sensible heat\\
$c_p$ & 1004 J kg$^{-1}$ K$^{-1}$ & specific heat of dry air \\
$\delta$ && declination \\
$e$ && vapor pressure in pascals \\
$e_s$ && saturation vapor pressure \\
$\epsilon$ & 0.622 & ratio of molecular weight of water to dry air \\
$H\!A$ && hour angle \\
$L$ & $2.5 \times 10^6$ J kg$^{-1}$ & latent heat of vaporization \\
$L$ & $2.834 \times 10^6$ J kg$^{-1}$ & latent heat of sublimation \\
$\phi$ && latitude \\
$Q_o$ && incoming radiation for cloudless skies \\
$q_s$ && surface specific humidity \\
$q_{10 \rm m}$ && 10 meter specific humidity \\
$\rho_a$ && air density \\
$S$ & 1353 W m$^{-2}$ & solar constant \\
$\sigma$ & $5.67 \times 10^{-8}$ W m$^{-2}$ K$^{-4}$ &
Stefan-Boltzmann constant \\
$T_a$ && air temperature \\
$T_d$ && dew point temperature \\
$T_{s\!f\!c}$ && surface temperature of the water/ice/snow \\
$V_{wg}$ && geostrophic wind speed \\
$Z$ && solar zenith angle \\

\hline
\end{tabular}
}
\caption{Variables used in computing the incoming radiation and latent
and sensible heat}
\label{radvars}
\end{table}

\subsection{Longwave radiation}

The clear sky formula for incoming longwave radiation is given by:
\begin{equation}
   F\!\downarrow\, = \sigma T_a^4 \left\{1 - 0.261 \exp \left[ -7.77 \times 10^{-4}
   (273 - T_a) ^2 \right] \right\}
\end{equation}
while the cloud correction is given by:
\begin{equation}
   LW\!\downarrow\, = (1 + 0.275 c)\, F\!\downarrow .
\end{equation}

\subsection{Sensible heat}

The sensible heat is given by the standard aerodynamic formula:
\begin{equation}
   H\!\downarrow\, = \rho_a c_p C_H V_{wg} (T_a - T_{s\!f\!c}) .
\end{equation}

\subsection{Latent heat}

The latent heat depends on the vapor pressure and the saturation vapor
pressure given by:
\begin{eqnarray}
   e & = & 611 \times 10^{a(T_d - 273.16) / (T_d - b)} \\
   e_s & = & 611 \times 10^{a(T_{s\!f\!c} - 273.16) / (T_{s\!f\!c} - b)}
\end{eqnarray}
The vapor pressures are used to compute specific humidities according
to:
\begin{eqnarray}
   q_{10 \rm m} & = & {\epsilon e \over p - (1 - \epsilon) e} \\
   q_s & = & {\epsilon e_s \over p - (1 - \epsilon) e_s}
\end{eqnarray}
The latent heat is also given by a standard aerodynamic formula:
\begin{equation}
   LE\!\downarrow\, = \rho_a L C_E V_{wg} (q_{10 \rm m} - q_s) .
\end{equation}
Note that these need to be computed independently for the ice-covered
and ice-free portions of each gridbox since the empirical factors
$a$ and $b$ and the factor $L$ differ depending on the surface type.
