\subsection{Vertical mixing schemes}
\label{Vmix}
SCRUM contains a variety of methods for setting the vertical viscous and
diffusive coefficients. The choices range from simply choosing fixed
values to the KPP and Mellor-Yamada turbulence closure schemes.
See \citet{Large98} for a review of surface ocean mixing schemes.
Many schemes have a background molecular value which is used when the
turbulent processes are assumed to be small (such as in the interior).

\subsubsection{Brunt-V\"ais\"al\"a frequency scheme}
One of the simplest schemes is to set the vertical diffusion
coefficients to be large when the water column is vertically unstable.
The vertical viscosity is uniform:
\begin{align}
   K_m &= K_{m_{\rm background}} \\
   K_s &=
   \begin{cases}
     C_o &                     \text{if $Ri_g < 0$,} \\
     K_{s_{\rm background}} & \text{if $Ri_g = 0$,} \\
     \min ( \nu_{\max}, \max ( \nu_{\min}, C / \sqrt{Ri_g} )) &
			       \text{if $Ri_g > 0$}.
   \end{cases}
\end{align}
Where $K_s$ applies to any scalar. The contants in this expression are
$C_o = 1.0$, $C = 10^{-7}$, $\nu_{\min} = 3 \times 10^{-5}$ and
$\nu_{\max} = 4 \times 10^{-4}$.

\subsubsection{Pacanowski-Philander}
\citet{Pacanowski81} developed a vertical
mixing parameterization based on measurements in the equatorial oceans:
\begin{align}
   K_m &= {\nu_o \over (1 + a Ri_g)^n} + K_{m_{\rm background}} \\
   K_s &= {K_m \over (1 + a Ri_g)} + K_{s_{\rm background}}
\end{align}
The contants here are $\nu_o = .01$, $n = 2$ and $a = 5$. The values of
$K_m$ and $K_s$ can get very large for negative values of $Ri_g$, so we
choose to limit these values to that obtained for $Ri_g = 0$.

\subsubsection{Mellor-Yamada}
One of the more popular closure schemes is that of 
\citet{Mellor74, Mellor82}. They actually present a hierarchy of
closures of increasing complexity. SCRUM provides only the
``Level 2.5'' closure with the \citet{Galperin88}
modifications as described in \citet{Allen95}.
This closure scheme adds two prognostic equations, one
for the turbulent kinetic energy (${1 \over 2} q^2$) and one for the
turbulent kinetic energy times a length scale ($q^2l$).

The turbulent kinetic energy equation is:
\begin{equation}
  {D \over Dt} \left( {q^2 \over 2} \right) -
  {\partial \over \partial z} \left[ K_q {\partial \over \partial z} 
  \left( {q^2 \over 2} \right) \right] = P_s + P_b - \xi_d
\end{equation}
where $P_s$ is the shear production, $P_b$ is the buoyant production
and $\xi_d$ is the dissipation of turbulent kinetic energy. 
These terms are given by
\begin{align}
   P_s &= K_m \left[ \left( {\partial u \over \partial z }\right)^2 +
   \left( {\partial v \over \partial z} \right)^2 \right],  \\
   P_b &= K_s N^2, \\
   \xi_d &= {q^3 \over B_1 l}
\end{align}
where $B_1$ is a constant.
One can also add a traditional horizontal Laplacian or biharmonic
diffusion (${\cal D}_q$) to the turbulent kinetic energy equation.
The form of this equation in the model coordinates becomes
{\samepage
\[
  {\partial \over \partial t} \left( {H_z q^2 \over mn} \right) +
  {\partial \over \partial \xi} \left( {H_z u q^2 \over n} \right) +
  {\partial \over \partial \eta} \left( {H_z v q^2 \over m} \right) +
  {\partial \over \partial s} \left( {H_z \Omega q^2 \over mn} \right) -
  {\partial \over \partial s} \left( {K_q \over mnH_z}
  {\partial q^2 \over \partial s} \right) =
\]
\begin{equation}
  {2H_z K_m \over mn} \left[ \left({\partial u \over \partial z}
  \right)^2 + \left( {\partial v \over \partial z} \right)^2 \right] +
  {2H_z K_s \over mn} N^2 - {2H_z q^3 \over mnB_1 l} +
  {H_z \over mn} {\cal D}_q .
\end{equation}
}
The vertical boundary conditions are:
\[
\begin{array}{rl}
  \mbox{top ($z = \zeta(x,y,t))$} \hspace{1cm}
  & {H_z \Omega \over mn} = 0 \\ [1.5mm]
  & {K_q \over mn H_z} \, \frac{\partial q^2}{\partial s}
    = {B_1^{2/3} \over \rho_o} \left[ \left( \tau_s^\xi \right)^2
    + \left( \tau_s^\eta \right)^2  \right] \\ [1.5mm]
  & H_z K_m \left( {\partial u \over \partial z},
    {\partial v \over \partial z} \right) = {1 \over \rho_o}
    \left( \tau_s^\xi, \tau_s^\eta \right) \\ [1.5mm]
  & H_z K_s N^2 = {Q \over \rho_o c_P} \\[2mm]
  \mbox{and bottom ($z = -h(x,y)$)} \hspace{1cm} &
    {H_z \Omega \over mn} = 0 \\[1.5mm]
  & {K_q \over mnH_z} {\partial q^2 \over \partial s}
    = {B_1^{2/3} \over \rho_o} \left[ \left( \tau_b^\xi \right)^2
    + \left( \tau_b^\eta \right)^2  \right] \\ [1.5mm]
  & H_z K_m \left( {\partial u \over \partial z},
    {\partial v \over \partial z} \right) = {1 \over \rho_o}
    \left( \tau_b^\xi, \tau_b^\eta \right) \\ [1.5mm]
  & H_z K_s N^2 = 0
\end{array}
\]
The equation is timestepped much like the model tracer equations,
including an implicit solve for the vertical operations.

There is also an equation for the turbulent length scale $l$:
\begin{equation}
  {D \over Dt} \left( {lq^2} \right) -
  {\partial \over \partial z} \left[ K_l
  {\partial lq^2 \over \partial z} 
  \right] = lE_1 ( P_s + P_b ) - {q^3 \over B_1} \tilde{W}
\end{equation}
where $\tilde{W}$ is the wall proximity function:
\begin{align}
  \tilde{W} &= 1 + E_2 \left( {l \over kL} \right) ^2 \\
  L^{-1} &= {1 \over \zeta -z} + {1 \over H+z}
\end{align}
The form of this equation in the model coordinates becomes
{\samepage
\[
  {\partial \over \partial t} \left( {H_z q^2l \over mn} \right) +
  {\partial \over \partial \xi} \left( {H_z u q^2l \over n} \right) +
  {\partial \over \partial \eta} \left( {H_z v q^2l \over m} \right) +
  {\partial \over \partial s} \left( {H_z \Omega q^2l \over mn} \right) -
  {\partial \over \partial s} \left( {K_q \over mnH_z}
  {\partial q^2l \over \partial s} \right) =
\]
\begin{equation}
  {H_z \over mn} lE_1 ( P_s + P_b) - 
  {H_z q^3 \over mnB_1 } \tilde{W} +
  {H_z \over mn} {\cal D}_{ql} .
\end{equation}
}
where ${\cal D}_{ql}$ is the horizontal diffusion of the quantity
$q^2l$.

Given these solutions for $q$ and $l$, the vertical viscosity and
diffusivity coefficients are:
\begin{align}
  K_m &= qlS_m + K_{m_{\rm background}} \\
  K_s &= qlS_h + K_{s_{\rm background}} \\
  K_q &= qlS_q + K_{q_{\rm background}}
\end{align}
and the stability coefficients $S_m$, $S_h$ and $S_q$ are found by
solving
\begin{equation}
  S_s \left[ 1 - (3A_2 B_2 + 18 A_1 A_2) G_h \right] =
  A_2 \left[ 1 - 6A_1 B_1^{-1} \right]
\end{equation}
\begin{equation}
  S_m \left[ 1 - 9A_1 A_2 G_h \right] - S_s \left[ G_h ( 18 A_1^2 +
  9A_1 A_2 ) G_h \right] =
  A_1 \left[ 1 - 3C_1 - 6A_1 B_1^{-1} \right]
\end{equation}
\begin{equation}
  G_h = \min ( -{l^2N^2 \over q^2}, 0.028 ).
\end{equation}
\begin{equation}
  S_q = 0.41 S_m
\end{equation}
The constants are set to $(A_1, A_2, B_1, B_2, C_1, E_1, E_2) = 
(0.92, 0.74, 16.6, 10.1, 0.08, 1.8, 1.33)$. The quantities $q^2$ and
$q^2l$ are both constrained to be no smaller than $10^{-8}$ while $l$
is set to be no larger than $0.53q/N$.
    
\subsection{The original Large, McWilliams and Doney parameterization}
\label{sec:origLMD}

The vertical mixing parameterization introduced by
\citet{Large94} is a versatile first order scheme which has
been shown to perform well in open ocean settings.  Its design
facilitates experimentation with additional or modified representations
of specific turbulent processes.

\subsubsection{Surface boundary layer}
The Large, McWilliams and Doney scheme (LMD)
matches separate parameterizations for vertical mixing
of the surface boundary layer and the ocean interior.  A formulation
based on boundary layer similarity theory is applied in the water
column above a calculated boundary layer depth $h_{sbl}$.  This
parameterization is then matched at the interior with schemes to
account for local shear, internal wave and double diffusive mixing
effects.  

Viscosity and diffusivities at model levels above a calculated
surface boundary layer depth ($h_{sbl}$ ) are expressed as the product
of the length scale $h_{sbl}$, a turbulent velocity scale $w_x$ and a
non-dimensional shape function.

\begin{equation}
\nu_x = h_{sbl} w_x(\sigma)G_x(\sigma)
\end{equation}
where $\sigma$ is a non-dimensional coordinate ranging from 0 to 1
indicating depth within the surface boundary layer. The $x$ subscript
stands for one of momentum, temperature and salinity.

\paragraph{Surface Boundary layer depth}
The boundary layer depth $h_{sbl}$ is calculated as the minimum of the
Ekman depth, estimated as,

\begin{equation}
h_e=0.7u_*/f
\end{equation}
(where $u_*$ is the friction velocity ($u_*=\sqrt{\tau_x^2+\tau_y^2}/\rho$ ),
 the Monin-Obukhov depth:
\begin{equation}
L=u_*^3/(\kappa B_f)
\end{equation}

\noindent and the shallowest depth at which a critical bulk Richardson
number is reached (typically between 0.25 and 0.5). The bulk Richardson
number, $Ri_b$ is calculated


%this one is the one that correctly makes the vectors bold but all the other letter
%spacing gets messed up when you make it so.    ???
%Ri_b(z)=\frac{(B_r-B(d))d}{|\mbox{\boldmath $V_r-V(d)$}|^2+{V_t}^2(d)}
% so instead...
\begin{equation}
Ri_b(z)=\frac{(B_r-B(d))d}{|V_r-V(d)|^2+{V_t}^2(d)}
\end{equation}
(where $d$ is distance down from the surface, $B$ is the buoyancy,
$B_r$ is the buoyancy at a near surface
reference depth, $V$ is the mean horizontal velocity, $V_r$ the
velocity at the near surface reference depth and $V_t$ is an estimate
of the turbulent velocity contribution to velocity shear.)

The turbulent velocity shear term in this equation is given by LMD as,

\begin{equation}
  V_{t}^{2}(d)=\frac{C_v(-\beta_T)^{1/2}}{Ri_c
  \kappa}(c_s\epsilon)^{-1/2}dNw_s
\end{equation}
LMD derive this term based on the expected behavior in the pure
convective limit.  The empirical rule of convection states that the
ratio of the surface buoyancy flux to that at the entrainment depth be 
a constant.  Thus the entrainment flux at the
bottom of the boundary layer under such conditions should be
independent of the stratification at that depth.  Without a turbulent
shear term in the denominator of the bulk Richardson number
calculation, the estimated boundary layer depth is too shallow and the
diffusivity at the entrainment depth is too low to obtain the
necessary entrainment flux.  Thus by adding a turbulent shear term
proportional to the stratification in the denominator, the calculated
boundary layer depth will be deeper and will lead to a high enough
diffusivity so that the empirical rule of convection is satisfied.
  
\paragraph{turbulent velocity scale}
To estimate $w_x$ (where $x$ is $m$ - momentum {\em or} $s$
- any scalar) throughout the boundary layer surface layer similarity
theory is utilized.  Following an argument by
\citet{TM86}, Large et al.\ estimate the velocity scale as

\begin{equation}
w_x=\frac{\kappa u_*}{\phi_x(\zeta)}
\end{equation}

where $\zeta$ is the surface layer stability parameter defined as
$z/L$.  $\phi_x$ is a non-dimensional flux profile which varies based
on the stability of the boundary layer forcing.  The stability
parameter used in this equation is assumed to vary over the entire
depth of the boundary layer in stable and neutral conditions.  In
unstable conditions it is assumed only to vary through the surface
layer which is defined as $ \epsilon h_{sbl} $ (where $\epsilon$ is
set at 0.10) .  Beyond this depth $\zeta$
is set equal to its value at $ \epsilon h_{sbl} $.

The flux profiles are expressed as analytical fits to atmospheric
surface boundary layer data.  In stable conditions they vary linearly
with the stability parameter $\zeta$  as

\begin{equation}
\phi_x=1+5\zeta
\end{equation}

In near-neutral unstable conditions common Businger-Dyer forms are
used which match with the formulation for stable conditions at
$\zeta=0$.  Near neutral conditions are defined as 

\begin{equation}
-0.2 \leq \zeta < 0
\end{equation}
for momentum and,
\begin{equation}
-1.0 \leq \zeta < 0
\end{equation}
for scalars.  The non dimensional flux profiles in this regime are,

\begin{equation}
\phi_m=(1-16\zeta)^{1/4}
\end{equation}

\begin{equation}
\phi_s=(1-16\zeta)^{1/2}
\end{equation}
In more unstable conditions $\phi_x$ is chosen to match the
Businger-Dyer forms and with the free convective limit.  Here the flux 
profiles are 

\begin{equation}
\phi_m=(1.26-8.38\zeta)^{1/3}
\end{equation}

\begin{equation}
\phi_s=(-28.86-98.96\zeta)^{1/3}
\end{equation}

\paragraph{The shape function}
The non-dimensional shape function $G(\sigma)$ is a third order
polynomial with coefficients chosen to match the interior viscosity at
the bottom of the boundary layer and Monin-Obukhov
similarity theory approaching the surface.  This function is defined
as a 3rd order polynomial.

\begin{equation}
G(\sigma)=a_o+a_1\sigma+a_2\sigma^2+a_3\sigma^3
\end{equation}
with the coefficients specified to match surface boundary conditions
and to smoothly blend with the interior,
\begin{equation}
a_o=0
\end{equation}
\begin{equation}
a_1=1
\end{equation}
\begin{equation}
a_2=-2+3\frac{\nu_{x}(h_{sbl})}{hw_x(1)}+\frac{\partial_x \nu_{x}(h)}{w_{x}(1)}+\frac{\nu_{x}(h)\partial_{\sigma}w_x(1)}{hw_{x}^{2}(1)}
\end{equation}
\begin{equation}
a_3=1-2\frac{\nu_{x}(h_{sbl})}{hw_x(1)}-\frac{\partial_x \nu_{x}(h)}{w_{x}(1)}-\frac{\nu_{x}(h)\partial_{\sigma}w_x(1)}{hw_{x}^{2}(1)}
\end{equation}
where $\nu_{x}(h)$ is the viscosity calculated by the interior
parameterization at the boundary layer depth.

\paragraph{Countergradient flux term}
The second term of the LMD scheme's surface boundary layer
formulation is the non-local transport term $\gamma$ which can play a
significant role in mixing during surface cooling events.  This is a
redistribution term included in the tracer equation separate from the
diffusion term and is written as 
\begin{equation}
-\frac{\partial}{\partial z}K\gamma.
\end{equation}

LMD base their formulation for non-local scalar transport on a
parameterization for pure free convection from
\citet{Mailhot82}. They extend this parameterization to cover any
unstable surface forcing conditions to give,
\begin{equation}
  \gamma_{T}=C_s\frac{\overline{wT_0}+
  \overline{wT_R}}{w_T(\sigma)h}
\end{equation}
for temperature and 
\begin{equation}
\gamma_S=C_s \frac{\overline{wS_0}}{w_S(\sigma)h}
\end{equation}
for salinity (other scalar quantities with surface fluxes can be
treated similarly). LMD argue that although there is evidence of
non-local transport of momentum as well, the form the term would take
is unclear so they simply specify $\gamma_m=0$.

\subsubsection{The interior scheme}
The interior scheme of Large, McWilliams and Doney estimates viscosity
coefficient by adding the effects of several generating mechanisms:
shear mixing, double-diffusive mixing and internal wave generated
mixing.
\begin{equation}
\nu_{x}(d)=\nu_{x}^s+\nu_{x}^d+\nu_{x}^w
\end{equation}

\paragraph{Shear generated mixing}
The shear mixing term is calculated using a
gradient Richardson number formulation, where the gradient Richardson
number is
\begin{equation}
  Ri_g = { {\partial B \over \partial z} \over
  \left({\partial U \over \partial z}\right)^2 +
  \left({\partial V \over \partial z}\right)^2 }
\end{equation}
with viscosity
estimated as: 

\begin{equation}
\nu_s=\begin{cases}
\nu_0&   \text{$ Ri_g<0$}, \\
\nu_0[1-(Ri_g/Ri_0)^2]^3&  \text{$0< Ri_g<Ri_0$},  \\
0&   \text{$Ri_g>Ri_0$}.  
\end{cases}
\end{equation}

where $\nu_0$ is $5.0 \times 10^{-3}$, $Ri_0 = 0.7$.  

\paragraph{Double diffusive processes}
The second component of the interior mixing parameterization represents
double diffusive mixing.  From limited sources of laboratory and field
data LMD parameterize the salt fingering case ($R_{\rho}>1.0$)

\begin{equation}
\nu_{s}^{d}(R_{\rho})=
	\begin{cases}
1\times10^{-4}[1-(\frac{(R_{\rho}-1}{R_{\rho}^0-1})^2)^{3}&   \text{for $1.0<R_{\rho}<R_{\rho}^0=1.9$},\\
           0& \text{otherwise}.
        \end{cases}
\end{equation}
\begin{equation}
\nu_{\theta}^{d}(R_{\rho})=0.7\nu_{s}^{d}
\end{equation}

For diffusive convection ($0<R_{\rho}<1.0$) LMD suggest several
formulations from the literature and choose the one with the most
significant impact on mixing \citep{Fedorov88}.

\begin{equation}
\nu_{\theta}^{d}=(1.5 \time 10^{-6})(0.909 \exp(4.6 \exp[-0.54(R_{\rho}^{-1}-1)])
\end{equation}
for temperature.  For other scalars,

\begin{equation}
   \nu_{s}^{d}=
	\begin{cases}
	     \nu_{\theta}^{d}(1.85-0.85R_{\rho}^{-1})R_{\rho}& \text{for $0.5<=R_{\rho}<1.0$},\\ 
             \nu_{\theta}^{d}0.15R_{\rho}&  \text{otherwise}. \\
        \end{cases}
\end{equation}

\paragraph{Internal wave generated mixing}
Internal wave generated mixing serves as the background mixing in the
LMD scheme.  It is specified as a constant for both scalars and
momentum.  Eddy diffusivity is estimated based on the data of
\citet{LWL93}.  While \citet{Peters88} suggest
eddy viscosity should be 7 to 10 times larger than diffusivity for
gradient Richardson numbers below approximately 0.7.  Therefore LMD use

\begin{equation}
\nu_{m}^w=1.0 \times 10^{-4} m^2 s^{-1}
\end{equation}
\begin{equation}
\nu_{s}^w=1.0 \times 10^{-5} m^2 s^{-1}
\end{equation}

