\section{Getting started}
\label{Starting}

\subsection{myroms.org}
\label{Svn}
Starting off with ROMS is not the easiest thing to do, and it just
seems to be getting more complex as time goes by. There are
some resources, however, beginning with the
electronic home for ROMS users at
\href{http://www.myroms.org}{www.myroms.org}. Go to register,
which gives you access to the subversion server for the code and to
the  discussion forum for all things ROMS. There is also a wiki,
a bug tracking system, and even a developer blog.

The wiki contains parts of this manual, but the nature of wikis is that
they can be more fluid, with more authors, than a static document such
as this. Dave Robertson (robertson@marine.rutgers.edu) is the one to
talk to if you care to contribute to the wiki.

\subsection{Prerequisites}
As mentioned in \S\ref{Introduction}, ROMS has some external
requirements. These are:
\begin{itemize}
    \item UNIX or UNIX-like environment, such as
    \href{http://www.redhat.com/services/custom/cygwin/}{Cygwin}.
    \item The
    \href{http://www.unidata.ucar.edu/software/netcdf/index.html}{NetCDF
    library}.
    \item \href{http://www.gnu.org/software/make/}{Gnu make} version
    3.81 or higher.
    \item \href{http://en.wikipedia.org/wiki/C_preprocessor}{C
    preprocessor}---the one from gnu with the \code{-traditional}
    flag works well.
\end{itemize}
\subsection{Acquiring the ROMS code}
The version of the model described in this document is a merger between
ROMS 3.2 and a sea-ice model. The main ROMS code is available for
download via \code{svn} at
\href{https://www.myroms.org/svn/src/}{https://www.myroms.org/svn/src/}.
The sea ice code is a branch off a different repository and requires
special access---contact Dave Robertson (as above) for more information.

\begin{itemize}
   \item You may be best served by going to the 
\href{https://www.myroms.org/wiki/}{ROMS wiki} which includes
sections called Getting Started and Tutorials.
   \item Don't be afraid to use the forum. It has everything from
employment opportunities to debugging help. Posting there can get
you help from one of several people, improving your odds of success
over private emails. Registered users get an email once a day about
new postings, so you might have to wait a day (or more) for a reply.
   \item There have been ROMS meetings and classes in which a tutorial
session is included as part of the program. There are various resources
from these online---I've heard good things about the
\href{http://eros.eas.gatech.edu/ROMS-Tutorial/tutorials.html}{tutorials}
from Manu Di Lorenzo.
   \item ROMS comes with several cases all ready to go at the flip
of a switch. Try these out first and get to understand how they
are set up.
    \begin{itemize}
      \item \S\ref{Build} describes how to pick the cases and set up
      the build environment.
      \item \S\ref{Cpp1} lists all the ROMS options that can be
      added to your case.
      \item \S\ref{Functionals} lists the fields which can be
      provided to ROMS via analytic expressions.
      \item \S\ref{ASCII_in} lists the input parameters ROMS reads
      from a text file at run time.
      \item \S\ref{Code} and \S\ref{Wave} are meant to be
      informative for the simple and not-so-simple cases. If that
      isn't the case, please let me know.
    \end{itemize}
\end{itemize}

\subsection{Warnings and bugs}
ROMS is not a large program by some standards, but it is still complex
enough to require some effort to use effectively.
Some specific things to be wary of include:
\begin{itemize}
  \item It is recommended that you use 64 bits of precision rather
than 32 bits.
  \item The code must be run through the C preprocessor before it
is compiled.  This can occasionally be dangerous, especially with
the newer ANSI C versions of \code{cpp}.  Potential problems are listed
in Appendix \ref{Cpp}. The gnu \code{cpp} with the \code{-traditional} flag
is known to work well.
  \item The vertical $\sigma$-coordinate was chosen as being a sensible
way to handle variations in the water depth as seen in the coastal
oceans. Changes to the code have allowed us to expand the well-behaved
range of depths and the range of values for \code{THETA\_S}, plus
there are some new vertical coordinate options I haven't tried. I am
therefore not comfortable with providing concrete limits as I have in
the past.
  \item Nevertheless, for realistic problems we often fail to resolve the
bathymetric slopes and we then resort to bathymetric smoothing.
This in turn changes the shape of the basin and leads to its own set of
problems, such as altered sill depths. Also, the currents will react to
the change in shelf slope---you are now solving a different problem.
You may want to explore a matlab tool for minimally smoothing
the bathymetry found at:

\href{http://www.liga.ens.fr/~dutour/Bathymetry/index.html}{http://www.liga.ens.fr/$\sim$dutour/Bathymetry/index.html}.
  \item There remain bugs in ROMS. If you find any, please report
them on the forum and/or the bug tracking system at myroms.org.
\end{itemize}

