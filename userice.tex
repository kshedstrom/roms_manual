\section{Description of the Ice Model and the Coupling}
\label{Userice}

\subsection{Ice model structure}
The flow of the main program for the ice model is shown in
Fig.\ \ref{fiflow}.  The overall structure essentially consists of two
components---the momentum equations and the ice continuity equations.
The momentum balance includes air and water stresses, Coriolis force,
internal ice stress, inertial forces and ocean tilt (equations
(\ref{ist1}) and (\ref{ist2})).  There is a choice of rheologies (form
of internal ice stress) including viscous-plastic (used here), free
drift, Mohr-Coulomb, and cavitating fluid.  A semi-implicit timestep is
done on the momentum equations which are solved by calling a solver
from the \code{NSPCG} library.  The ice viscosities are non-linear
functions of the velocity, so the solution is iterated several times,
with the viscosities being recomputed each iteration.

The ice continuity consists of advection (done in \code{ice\_mpdata} or
\code{ice\_advect}), and thermodynamics (\code{run\_mk} and
\code{ice\_mk}).
%Paul Budgell found that he could leave out the
%diffusivity of ice thickness and concentration if he used an advection
%scheme that preserved minima and maxima and did not lead to ringing
%(\cite{Smolark90} and \cite{Smolark93}).
We have also added the diffusion of the ice tracer variables in
order to obtain a smoother solution. This may be optional if the
forcing fields are sufficiently smooth.

\begin{figure}[p]
\begin{center}
\setlength{\unitlength}{3947sp}%
%
\begin{picture}(6399,9174)(1714,-9373)
\thinlines
\put(4814,-6608){\line(-5,-3){750}}
\put(4064,-7058){\line( 5,-3){750}}
\put(4814,-7508){\line( 5, 3){750}}
\put(5564,-7058){\line(-5, 3){750}}
\put(4814,-2258){\line(-5,-3){750}}
\put(4064,-2708){\line( 5,-3){750}}
\put(4814,-3158){\line( 5, 3){750}}
\put(5564,-2708){\line(-5, 3){750}}
\put(4201,-2011){\framebox(1200,450){}}
\put(4201,-4711){\framebox(1200,450){}}
\put(4801,-6436){\line( 1, 0){2925}}
\put(7726,-6436){\line( 0, 1){3750}}
\put(7726,-2686){\vector(-1, 0){2172}}
\put(4801,-9061){\line( 0,-1){300}}
\put(4801,-9361){\line( 1, 0){3300}}
\put(8101,-9361){\line( 0, 1){8250}}
\put(8101,-1111){\vector(-1, 0){3300}}
\put(4801,-661){\vector( 0,-1){900}}
\put(4801,-2011){\vector( 0,-1){249}}
\put(4606,-3055){\vector(-3,-2){450}}
\put(4993,-3055){\vector( 3,-2){450}}
\put(6813,-3814){\vector(-4,-1){1784}}
\put(5476,-3812){\vector(-4,-3){596}}
\put(4367,-2898){\vector(-4,-1){1812}}
\put(2527,-3803){\vector( 4,-1){1800}}
\put(4801,-4711){\vector( 0,-1){300}}
\put(4167,-3815){\vector( 1,-1){443}}
\put(6376,-3812){\framebox(1200,450){}}
\put(3076,-3811){\framebox(1650,450){}}
\put(1726,-3811){\framebox(1200,450){}}
\put(4201,-5461){\framebox(1200,450){}}
\put(3901,-6211){\framebox(1800,450){}}
\put(4801,-6211){\vector( 0,-1){375}}
\put(4801,-5461){\vector( 0,-1){300}}
\put(5994,-8153){\vector(-2,-1){900}}
\put(3593,-8160){\vector( 2,-1){900}}
\put(4876,-3811){\framebox(1350,450){}}
\put(3001,-8161){\framebox(1200,450){}}
\put(5401,-8161){\framebox(1200,450){}}
\put(4469,-7307){\vector(-3,-2){591}}
\put(5138,-7311){\vector( 3,-2){573}}
\put(4201,-661){\framebox(1200,450){}}
\put(4201,-9061){\framebox(1200,450){}}
\put(5223,-2916){\vector( 4,-1){1720}}
\put(4801,-511){\makebox(0,0)[b]{\code{ice\_init}}}
\put(4801,-1861){\makebox(0,0)[b]{\code{ice\_forcing}}}
\put(4801,-4561){\makebox(0,0)[b]{\code{ice\_gencoef}}}
\put(4801,-2761){\makebox(0,0)[b]{{rheology?}}}
\put(6751,-2611){\makebox(0,0)[b]{{iterate 3 times}}}
\put(6976,-3661){\makebox(0,0)[b]{\code{ice\_freedrift}}}
\put(5551,-3661){\makebox(0,0)[b]{\code{ice\_cavitating}}}
\put(2326,-3661){\makebox(0,0)[b]{\code{ice\_viscplast}}}
\put(3901,-3661){\makebox(0,0)[b]{\code{ice\_mohrcoulomb}}}
\put(4801,-5311){\makebox(0,0)[b]{\code{ice\_rhs}}}
\put(4801,-6061){\makebox(0,0)[b]{\code{ice\_solver\_NSPCG}}}
\put(4801,-7036){\makebox(0,0)[b]{{advection}}}
\put(4801,-7186){\makebox(0,0)[b]{{scheme?}}}
\put(3601,-8011){\makebox(0,0)[b]{\code{ice\_mpdata}}}
\put(4801,-8911){\makebox(0,0)[b]{\code{run\_mk}}}
\put(6001,-8011){\makebox(0,0)[b]{\code{ice\_advect}}}
\put(3736,-7471){\makebox(0,0)[b]{{MPDATA}}}
\put(6091,-7456){\makebox(0,0)[b]{{3rd-order upwind}}}
\end{picture}
\end{center}
\caption{Flow chart for the sea-ice model.}
\label{fiflow}
\end{figure}

The main program calls \code{ice\_step}, which in turn calls:
\begin{klist}
  \kitem{ice\_advect}  Compute the ice advection according to a
third-order upwind advection scheme. The ice ridging and ice diffusion
also happen here if they are enabled.
  \kitem{ice\_bctrans}  Boundary conditions for $A$, $h_i$ and $h_s$.
  \kitem{ice\_bcuv}  Boundary conditions for ice $u$ and $v$.
  \kitem{ice\_cavitating}  Compute the ice viscosities according to a
cavitating fluid rheology.
  \kitem{ice\_freedrift} Compute the ice parameters to model free
drift (no internal ice stress contribution).
  \kitem{ice\_gencoef}  Generate coefficients for the iterative
solution to the ice momentum equations.
  \kitem{ice\_mohrcoulomb}  Compute the ice viscosities according to a
Mohr-Coulomb fluid rheology.
  \kitem{ice\_mpdata}  Compute the advection of a scalar according to
a Smolarkiewicz advection scheme.
  \kitem{ice\_rhs}  Gather the right-hand-side terms for the ice
momentum equations prior to using the solver.
  \kitem{ice\_solver\_NSPCG}  Solve the implicit ice momentum equations
using the \code{NSPCG} library.
  \kitem{run\_mk}  Compute the changes in ice thickness and
concentration due to the thermodynamics.  Also produce the surface heat
and salt forcing for the ocean model.
  \kitem{ice\_viscplast}  Compute the ice viscosities according to a
viscous-plastic fluid rheology.
\end{klist}

\subsubsection{Thermodynamic subroutines}
The thermodynamic subroutines used in the model are:
\begin{klist}
  \kitem{ice\_mk}  Compute the net energy budget and change of
thickness for ice and snow.
  \kitem{ice\_frazil}  Compute the frazil ice growth due to supercooled
water in the ocean (called from \code{main3d} after the ocean timestep).
  \kitem{rads}     Compute the heat flux budgets over the water.
\end{klist}

\subsubsection{Initialization}
During initialization, the following routines are called:
\begin{klist}
  \kitem{def\_ice\_avg}  Create the netCDF ice averages file.
  \kitem{def\_ice\_his}  Create the netCDF ice history file.
  \kitem{def\_ice\_rst}  Create the netCDF ice restart file.
  \kitem{freeze}    Make sure that no water temperatures are below
freezing during initialization. It does not form frazil ice.
  \kitem{get\_ice}  Read a record from the netCDF ice restart file.
  \kitem{hakkblkdat}  Block data initializing some parameters for the
ice thermodynamics.
  \kitem{ice\_blkdat}  Block data initializing some parameters for the
Smolarkiewicz advection routine.
  \kitem{ice\_init} Initialize the ice model, either by reading a
history file or by setting the initial values.
  \kitem{ice\_user1}  Initialize some ice variables.
  \kitem{init\_hakkis}  Initialize some ice thermodynamic variables.
  \kitem{init\_ice}  Initialize some more ice variables.
\end{klist}

\subsubsection{Forcing fields} The ice model requires some extra
forcing fields. These fields require their own routines for reading
them from the netCDF forcing files:
\begin{klist}
     \kitem{get\_airp}  Reads surface air pressure
   from the forcing NetCDF file, and then linearly
   time-interpolates to current model time.
     \kitem{get\_airt}  Reads surface air temperature
   from the forcing NetCDF file, and then linearly
   time-interpolates to current model time.
     \kitem{get\_cloud}  Reads cloud fraction
   from the forcing NetCDF file, and then linearly
   time-interpolates to current model time.
     \kitem{get\_dewt}  Reads surface dewpoint temperature
   from the forcing NetCDF file, and then linearly
   time-interpolates to current model time.
     \kitem{get\_lrflux}  Reads incoming longwave radiation
   from the forcing NetCDF file, and then linearly
   time-interpolates to current model time.
     \kitem{get\_precip}  Reads precipitation
   from the forcing NetCDF file, and then linearly
   time-interpolates to current model time.
     \kitem{get\_wind}  Reads surface winds
   from the forcing NetCDF file, and then linearly
   time-interpolates to current model time.
\end{klist}


\subsubsection{Other subroutines}
The other subroutines in the ice model include:
\begin{klist}
  \kitem{ice\_forcing}  Computes the forcing fields due to exchange of
momentum by the ice and ocean.
  \kitem{smooth} Smooths a field with a five-point Laplacian filter.
  \kitem{wrt\_ice\_avg} Writes a record to the netCDF ice averages file.
  \kitem{wrt\_ice\_his} Writes a record to the netCDF ice history file.
  \kitem{wrt\_ice\_rst} Writes a record to the netCDF ice restart file.
\end{klist}

\subsection{Coupling strategy}
\label{Coupled}
The flow chart for the coupled ice-ocean model is shown in Fig.\
\ref{fboth}.  The ice model is stepped first since it provides the
surface heat and salt fluxes to the ocean model.  After the ocean model
is timestepped, the ocean temperatures are checked for supercooled
water---if any is found it is converted into frazil ice and the ocean
temperature and salinity are adjusted to be at freezing.  The ocean
equation of state is computed after this conversion.

\begin{figure}[p]
\begin{center}
\setlength{\unitlength}{3947sp}%
%
\begin{picture}(2424,8424)(3889,-8923)
\thinlines
\put(4201,-5911){\framebox(1200,450){}}
\put(4201,-6811){\framebox(1200,450){}}
\put(4201,-7711){\framebox(1200,450){}}
\put(4201,-8611){\framebox(1200,450){}}
\put(4801,-5011){\vector( 0,-1){450}}
\put(4801,-5911){\vector( 0,-1){450}}
\put(4801,-6811){\vector( 0,-1){450}}
\put(4801,-7711){\vector( 0,-1){450}}
\put(4801,-4111){\vector( 0,-1){450}}
\put(4801,-8611){\line( 0,-1){300}}
\put(4801,-8911){\line( 1, 0){1500}}
\put(6301,-8911){\line( 0, 1){6600}}
\put(6301,-2311){\vector(-1, 0){1500}}
\put(4201,-1861){\framebox(1200,450){}}
\put(4201,-961){\framebox(1200,450){}}
\put(4801,-1861){\vector( 0,-1){900}}
\put(4801,-961){\vector( 0,-1){450}}
\put(4201,-4111){\framebox(1200,450){}}
\put(4801,-3211){\vector( 0,-1){450}}
\put(4201,-3211){\framebox(1200,450){}}
\put(3901,-5011){\framebox(1800,450){}}
\put(4801,-811){\makebox(0,0)[b]{\code{initial}}}
\put(4801,-1711){\makebox(0,0)[b]{\code{ice\_init}}}
\put(4801,-3061){\makebox(0,0)[b]{\code{set\_vbc}}}
\put(4801,-3961){\makebox(0,0)[b]{\code{ice\_step}}}
\put(4801,-4861){\makebox(0,0)[b]{{3-D right-hand-sides}}}
\put(4801,-5761){\makebox(0,0)[b]{{2-D loop}}}
\put(4801,-6661){\makebox(0,0)[b]{\code{step3d}}}
\put(4801,-7561){\makebox(0,0)[b]{\code{ice\_frazil}}}
\put(4801,-8461){\makebox(0,0)[b]{\code{rho\_eos}}}
\end{picture}
\end{center}
\caption{Flow chart for the coupled ice-ocean model.}
\label{fboth}
\end{figure}

\subsection{C preprocessor variables}
The ice model has two C preprocessor variables in \code{cppdefs.h}:
\begin{klist}
  \kitem{ICE} Define to use ice component of the model.
  \kitem{ICE\_THERMO} Define for ice thermodynamics.
\end{klist}

The ice model requires new forcing fields to be read or computed:
\begin{klist}
  \kitem{ANA\_AIRT}   Define for an analytic air temperature.
  \kitem{ANA\_CLOUD}  Define for an analytic cloud fraction.
  \kitem{ANA\_DEWT}   Define for an analytic dew-point temperature.
  \kitem{ANA\_LRFLUX} Define for an analytic incoming longwave
radiation.
  \kitem{ANA\_SNOW}   Define for an analytic snow fall rate.
  \kitem{ANA\_SLP}    Define for an analytic sea-level pressure.
  \kitem{ANA\_WIND}   Define for analytic wind fields.
\end{klist}

There are also some C preprocessor variables in \code{icedefs.h}:
\begin{klist}
   \kitem{NSPCG} Use the \code{nspcg} library for the implicit solver.
Either this or \code{ESSL} must be turned on.
   \kitem{ESSL} Use the \code{essl} library for the implicit solver.
   \kitem{ANA\_ICE\_INIT} Initialize the ice fields analytically as
opposed to reading them from a file.
   \kitem{ICE\_MOMENTUM} Compute the ice momentum equations.
   \kitem{ICE\_ADVECT} Advect the ice tracers.
   \kitem{ice\_GSCHEME}  Use a third-order upwind advection scheme for
the tracers. This must be defined for either of these to take effect:
    \begin{klist}
       \kitem{ICE\_DIFFUSION} Add a Laplacian diffusion on the ice
     tracers.
%       \kitem{ICE\_RIDGING} Add an ice ridging scheme from \cite{Gray95}.
    \end{klist}
\end{klist}
