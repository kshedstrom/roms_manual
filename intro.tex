\setcounter{page}{1}
\section{Introduction}
This user's manual for the Regional Ocean Modeling System (ROMS)
describes the model equations and algorithms, as well as
additional user configurations necessary for specific applications.
This manual also describes the sea-ice model
that we are using (Budgell \cite{Budgell05}).

The principle attributes of the model are as follows:
\begin{klist}
\kitem{General} \mbox{}
\begin{itemize}
  \item Primitive equations with potential temperature, salinity, and an
equation of state.
  \item Hydrostatic and Boussinesq approximations.
  \item Optional third-order upwind advection scheme.
  \item Optional Smolarkiewicz advection scheme for tracers
    (potential temperature, salinity, etc.).
% \item Second-moment conservation.
  \item Optional Lagrangian floats.
  \item Option for point sources and sinks.
\end {itemize}
\kitem{Horizontal} \mbox{}
\begin{itemize}
  \item Orthogonal-curvilinear coordinates.
  \item Arakawa C grid.
  \item Closed basin, periodic, prescribed, radiation, and
    gradient open boundary conditions.
  \item Masking of land areas.
\end {itemize}
\kitem{Vertical} \mbox{}
\begin{itemize}
  \item $\sigma$ (terrain-following) coordinate.
  \item Free surface.
  \item Tridiagonal solve with implicit treatment of vertical
    viscosity and diffusivity.
\end {itemize}
\kitem{Ice} \mbox{}
\begin{itemize}
  \item Hunke and Dukowicz elastic-viscous-plastic dynamics.
  \item Mellor-Kantha thermodynamics.
  \item Orthogonal-curvilinear coordinates.
  \item Arakawa C grid.
  \item Smolarkiewicz advection of tracers.
%  \item Optional ridging scheme.
\end{itemize}
\kitem{Mixing options} \mbox{}
\begin{itemize}
  \item Horizontal Laplacian and biharmonic
    diffusion along constant $s$, $z$ or density
    surfaces.
  \item Horizontal Laplacian and biharmonic viscosity
    along constant $s$ or $z$ surfaces.
  \item Optional Smagorinsky horizontal viscosity and diffusion (but
  not recommended for diffusion).
%  \item Optional Gent and McWilliams \cite{Gent90}, \cite{Gent95} 
%    eddy-induced horizontal mixing.
  \item Horizontal free-slip or no-slip boundaries.
  \item Vertical harmonic viscosity and diffusion with a spatially
    variable coefficient, with options to compute the coefficients
    with Large et al.\ \cite{Large94}, Mellor-Yamada \cite{Mellor74},
    or generic length scale (GLS) \cite{Umlauf2001} mixing schemes.
\end{itemize}
\kitem{Implementation} \mbox{}
\begin{itemize}
  \item Dimensional in meter, kilogram, second (MKS) units.
  \item Fortran 90.
  \item Runs under UNIX, requires the C preprocessor, gnu make, and
  Perl.
  \item All input and output is done in NetCDF \cite{netCDF} (Network
    Common Data Format), requires the NetCDF library.
  \item Options include serial, parallel with MPI, and parallel with
  OpenMP.
%  \item Pre- and post-processing graphics package available which
%    uses the NCAR (National Center for Atmospheric Research)
%    graphics libraries.
\end{itemize}
\end{klist}
The above list hasn't changed so very much in the past ten to fifteen
years, but many of the numerical details have changed a great deal.
Examples include consistent temporal averaging of the barotropic
mode to guarantee both exact conservation and constancy preservation
properties for tracers; redefined barotropic pressure-gradient terms
to account for local variations in the density field; vertical
interpolation performed using conservative parabolic splines; and
higher-order, quasi-monotone advection algorithms.

ROMS now comes with a full suite of advanced data assimilation
routines; these options are beyond the scope of this document.

Chapters \ref{Phys} and \ref{Num} describe the model physics and
numerical techniques and contain information from Shchepetkin and
McWilliams \cite{SS2008b} and Haidvogel et al.\
\cite{Haidvogel07}.
Chapter \ref{Iphys} describes the ice equations and
Chapter \ref{Code} lists the model subroutines and functions.
%Chapter \ref{Progs} describes
%the support programs which are needed to provide ROMS with data files.
As distributed, ROMS is ready to run with a number of example problems.
The process of configuring ROMS for a particular application and
running it is
described in Chapter \ref{Wave}, including a discussion of a few example
applications.
%Chapter \ref{Floats} describes the Lagrangian floats and
%how to use them, including the plotting postprocessor \code{fltplt}.
Chapter \ref{Plothist} describes Hernan Arango's plotting
programs \code{cnt}, \code{ccnt}, \code{sec}, and \code{csec}.
%while chapter
%\ref{Userice} describes the ice subroutines and the coupling procedure.

\subsection{myroms.org}
\label{Svn}
We maintain an electronic home for ROMS users at
\href{http://www.myroms.org}{www.myroms.org}. Go to register,
which gives you access to the subversion server for the code and to
the  discussion forum for all things ROMS. There is also a wiki,
a bug tracking system, and even a developer blog.

The wiki contains parts of this manual, but the nature of wikis is that
they can be more fluid, with more authors, than a static document such
as this. Dave Robertson (robertson@marine.rutgers.edu) is the one to
talk to if you care to contribute to the wiki.

\subsubsection{Acquiring the ROMS code}
The version of the model described in this document is a merger between
ROMS 3.2 and a sea-ice model. The main ROMS code is available for
download via \code{svn} at
\href{https://www.myroms.org/svn/src/}{https://www.myroms.org/svn/src/}.
The sea ice code is a branch off a different repository and requires
special access---contact Dave Robertson (as above) for more information.

\subsection{Getting started}
Starting off with ROMS is not the easiest thing to do. There are
some resources, however:
\begin{itemize}
   \item You may be best served by going to the 
\href{https://www.myroms.org/wiki/}{ROMS wiki} which includes
sections called Getting Started and Tutorials.
   \item Don't be afraid to use the forum. It has everything from
employment opportunities to debugging help. Posting there can get
you help from one of several people, improving your odds of success
over private emails. Registered users get an email once a day about
new postings, so you might have to wait a day (or more) for a reply.
   \item There have been ROMS meetings and classes in which a tutorial
session is included as part of the program. There are various resources
from these online---I've heard good things about the
\href{http://eros.eas.gatech.edu/ROMS-Tutorial/tutorials.html}{tutorials}
from Manu Di Lorenzo.
   \item ROMS comes with several cases all ready to go at the flip
of a switch. Try these out first and get to understand how they
are set up.
    \begin{itemize}
      \item \S\ref{Build} describes how to pick the cases and set up
      the build environment.
      \item \S\ref{Cpp1} lists all the ROMS options that can be
      added to your case.
      \item \S\ref{Functionals} lists the fields which can be
      provided to ROMS via analytic expressions.
      \item \S\ref{ASCII_in} lists the input parameters ROMS reads
      from a text file at run time.
      \item \S\ref{Code} and \S\ref{Wave} are meant to be
      informative for the simple and not-so-simple cases. If that
      isn't the case, please let me know.
    \end{itemize}
\end{itemize}

\subsection{Warnings and bugs}
ROMS is not a large program by some standards, but it is still complex
enough to require some effort to use effectively.
Some specific things to be wary of include:
\begin{itemize}
  \item It is recommended that you use 64 bits of precision rather
than 32 bits.
  \item The code must be run through the C preprocessor before it
is compiled.  This can occasionally be dangerous, especially with
the newer ANSI C versions of \code{cpp}.  Potential problems are listed
in Appendix \ref{Cpp}. The gnu \code{cpp} with the \code{-traditional} flag
is known to work well.
  \item The vertical $\sigma$-coordinate was chosen as being a sensible
way to handle variations in the water depth as seen in the coastal
oceans. Changes to the code have allowed us to expand the well-behaved
range of depths and the range of values for \code{THETA\_S}, plus
there are some new vertical coordinate options I haven't tried. I am
therefore not comfortable with providing concrete limits as I have in
the past.
  \item Nevertheless, for realistic problems we often fail to resolve the
bathymetric slopes and we then resort to bathymetric smoothing.
This in turn changes the shape of the basin and leads to its own set of
problems, such as altered sill depths. Also, the currents will react to
the change in shelf slope---you are now solving a different problem.
You may want to explore a matlab tool for minimally smoothing
the bathymetry found at:

\href{http://www.liga.ens.fr/~dutour/Bathymetry/index.html}{http://www.liga.ens.fr/$\sim$dutour/Bathymetry/index.html}.
  \item There remain bugs in ROMS. If you find any, please report
them on the forum and/or the bug tracking system at myroms.org.
\end{itemize}
