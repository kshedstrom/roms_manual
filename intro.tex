\setcounter{page}{1}
\section{Introduction}
This user's manual for the Regional Ocean Modeling System (ROMS)
describes the model equations and algorithms, as well as
additional user configurations necessary for specific applications.
This manual also describes the sea-ice model
that we are using (Budgell \cite{Budgell05}).

The principle attributes of the model are as follows:
\begin{klist}
\kitem{General} \mbox{}
\begin{itemize}
  \item Primitive equations with potential temperature, salinity, and an
equation of state.
  \item Hydrostatic and Boussinesq approximations.
  \item Optional third-order upwind advection scheme.
  \item Optional Smolarkiewicz advection scheme for tracers
    (potential temperature, salinity, etc.).
% \item Second-moment conservation.
  \item Optional Lagrangian floats.
  \item Option for point sources and sinks.
\end {itemize}
\kitem{Horizontal} \mbox{}
\begin{itemize}
  \item Orthogonal-curvilinear coordinates.
  \item Arakawa C grid.
  \item Closed basin, periodic, prescribed, radiation, and
    gradient open boundary conditions.
  \item Masking of land areas.
\end {itemize}
\kitem{Vertical} \mbox{}
\begin{itemize}
  \item $\sigma$ (terrain-following) coordinate.
  \item Free surface.
  \item Tridiagonal solve with implicit treatment of vertical
    viscosity and diffusivity.
\end {itemize}
\kitem{Ice} \mbox{}
\begin{itemize}
  \item Hunke and Dukowicz elastic-viscous-plastic dynamics.
  \item Mellor-Kantha thermodynamics.
  \item Orthogonal-curvilinear coordinates.
  \item Arakawa C grid.
  \item Smolarkiewicz advection of tracers.
%  \item Optional ridging scheme.
\end{itemize}
\kitem{Mixing options} \mbox{}
\begin{itemize}
  \item Horizontal Laplacian and biharmonic
    diffusion along constant $s$, $z$ or density
    surfaces.
  \item Horizontal Laplacian and biharmonic viscosity
    along constant $s$ or $z$ surfaces.
  \item Optional Smagorinsky horizontal viscosity and diffusion (but
  not recommended for diffusion).
%  \item Optional Gent and McWilliams \cite{Gent90}, \cite{Gent95} 
%    eddy-induced horizontal mixing.
  \item Horizontal free-slip or no-slip boundaries.
  \item Vertical harmonic viscosity and diffusion with a spatially
    variable coefficient, with options to compute the coefficients
    with Large et al.\ \cite{Large94}, Mellor-Yamada \cite{Mellor74},
    or generic length scale (GLS) \cite{Umlauf2001} mixing schemes.
\end{itemize}
\kitem{Implementation} \mbox{}
\begin{itemize}
  \item Dimensional in meter, kilogram, second (MKS) units.
  \item Fortran 90.
  \item Runs under UNIX, requires the C preprocessor, gnu make, and
  Perl.
  \item All input and output is done in NetCDF \cite{netCDF} (Network
    Common Data Format), requires the NetCDF library.
  \item Options include serial, parallel with MPI, and parallel with
  OpenMP.
%  \item Pre- and post-processing graphics package available which
%    uses the NCAR (National Center for Atmospheric Research)
%    graphics libraries.
\end{itemize}
\end{klist}
The above list hasn't changed so very much in the past ten to fifteen
years, but many of the numerical details have changed a great deal.
Examples include consistent temporal averaging of the barotropic
mode to guarantee both exact conservation and constancy preservation
properties for tracers; redefined barotropic pressure-gradient terms
to account for local variations in the density field; vertical
interpolation performed using conservative parabolic splines; and
higher-order, quasi-monotone advection algorithms.

ROMS now comes with a full suite of advanced data assimilation
routines; these options are beyond the scope of this document.

Chapters \ref{Phys} and \ref{Num} describe the model physics and
numerical techniques and contain information from Shchepetkin and
McWilliams \cite{SS2008b} and Haidvogel et al.\
\cite{Haidvogel07}.
Chapter \ref{Code} lists the model subroutines and functions.
%Chapter \ref{Progs} describes
%the support programs which are needed to provide ROMS with data files.
As distributed, ROMS is ready to run with a number of example problems.
The process of configuring ROMS for a particular application and
running it is
described in Chapter \ref{Wave}, including a discussion of a few example
applications.
%Chapter \ref{Floats} describes the Lagrangian floats and
%how to use them, including the plotting postprocessor \code{fltplt}.
Chapter \ref{Plothist} describes Hernan Arango's plotting
programs \code{cnt}, \code{ccnt}, \code{sec}, and \code{csec}.
Chapter \ref{Iphys} describes the ice equations while chapter
\ref{Userice} describes the ice subroutines and the coupling procedure.

\subsection{Acquiring the ROMS code}
\label{Svn}
The version of the model described in this document is a merger between
ROMS 3.2 and a sea-ice model. The main ROMS code is available for
download via \code{svn} once you have a login on the ROMS site. The sea ice
code is a branch off of that and requires special access - contact
Dave Robertson (robertson@marine.rutgers.edu) for more information.

\subsection{The ROMS forum}
We maintain an electronic home for ROMS users at www.myroms.org,
with a subversion server for the code. There is also a discussion
forum for all things ROMS, from jobs to debugging help, plus a
bug tracking system, and a wiki too. The wiki is given parts of
this manual as they are created, but the nature of wikis is that
they can be more fluid, with more authors, than a static document such
as this. Again, Dave Robertson is the one to talk to if you care to
contribute to the wiki.

\subsection{Warnings and bugs}
ROMS is not a large program by some standards, but it is still complex
enough to require some effort to use effectively.  Section \ref{User}
attempts to describe what the user is responsible for---please read
it carefully.

More specific things to be wary of include:
\begin{itemize}
  \item It is recommended that you use 64 bits of precision rather
than 32 bits.
  \item The code must be run through the C preprocessor before it
is compiled.  This can occasionally be dangerous, especially with
the newer ANSI C versions of \code{cpp}.  Potential problems are listed
in Appendix \ref{Cpp}. The gnu \code{cpp} with the \code{-traditional} flag
is known to work well.
  \item The vertical $\sigma$ coordinate was chosen as being a sensible
way to handle variations in the water depth.  It has been used with
success when the maximum and minimum depths differ by a factor of
twenty or less, and the value of the stretching parameter,
\code{THETA\_S}, is between zero and five.  It is also desirable to
have the depth variations be well resolved by the
horizontal grid.  For realistic problems we often fail to resolve the
bathymetric slopes and we then resort to bathymetric smoothing.
This in turn changes the shape of the basin and leads to its own set of
problems, such as altered sill depths. Also, the currents will react to
the change in shelf slope---you are now solving a different problem.
\end{itemize}
