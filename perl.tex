\section{\code{Perl} scripts for Fortran}
\label{Perl}
\code{Perl} is a computer language, invented by Larry Wall, for
manipulating text and other useful things.  It is fully described in
\citet{perl1}, while a more tutorial approach is given
by \citet{perl2}.  \code{Perl} itself is available from your
nearest \code{CPAN} archive site, for instance:
\begin{verbatim}
      http://www.perl.com/CPAN/
\end{verbatim}

I have several \code{Perl} scripts which I find useful when working
with Fortran programs, and which are available from:
\begin{verbatim}
      http://marine.rutgers.edu/po/perl.html
\end{verbatim}
It is not necessary to know \code{Perl} to
use these scripts, but it must be installed on your system.  To use
these scripts, simply place them somewhere in your path and make sure
that they are executable:
\begin{verbatim}
      chmod 755 relabel
\end{verbatim}
(on a UNIX machine).

The following scripts modify your source code and usually work on the
style of Fortran in SCRUM, but have been known to do the wrong thing.
Some of the scripts become confused when part of an \code{if} or
\code{do} statement is inside an \code{\#ifdef} clause.  The following
will parse as two nested do loops, only one of which is terminated:
\begin{verbatim}
     #ifdef EW_PERIODIC
           do i=1,Lm
     #else
           do i=0,L
     #endif
             :
           enddo
\end{verbatim}
It is also extremely dangerous to run \code{relabel} on an arithmetic
\code{if}.

{\bf Do not delete your original code before checking the new code.}

\subsection{\code{redo}}
This program reformats \code{do} loops and was written to convert
\begin{verbatim}
          do 10 i=1,20
      10     sum = sum+i
\end{verbatim}
to
\begin{verbatim}
          do 10 i=1,20
             sum = sum+i
      10  continue
\end{verbatim}
The \code{-E} option tells it to use \code{enddo} instead of
\code{continue} as in
\begin{verbatim}
          do  i=1,20
             sum = sum+i
          enddo
\end{verbatim}
\code{redo} is used as follows:
\begin{verbatim}
       redo < file.F > file.new
\end{verbatim}
\code{redo} was written so that \code{findent} would work on SPEM.

\subsection{\code{findent}}
\code{findent} will indent your Fortran code two spaces for \code{do}
loops and \code{if} statements.  It will not correct lines which extend
beyond 72 characters, but will print out a warning for each one.  It
assumes that each \code{do} loop ends with a \code{continue} or
\code{enddo}.  \code{findent} is used as follows:
\begin{verbatim}
       findent < file.F > file.new
\end{verbatim}
There is an option to change the number of spaces for each level of
indenting.  To get an indent of four instead of two, use:
\begin{verbatim}
       findent -n 4 < file.F > file.new
\end{verbatim}
See the comments at the top of the code for the more obscure options.

\subsection{\code{relabel}}
\code{relabel} was written by Sverre Froyen to replace the numbered
Fortran labels with new sequentially ordered labels.  It was the first
of these scripts and helped me to write the rest.  \code{relabel} is
used as follows:
\begin{verbatim}
       relabel < file.F > file.new
\end{verbatim}
It does have some known bugs, however:
\begin{itemize}
\item No computed goto.
\item No assigned goto or assign.
\item No arithmetic if.
\item No new-lines inside the parenthesis immediately following a
read/write.
\item Others not yet discovered.
\end{itemize}

All the source files in SPEM have been run through \code{redo},
\code{findent}, and \code{relabel}.  In the C shell (\code{csh}), a
series of files can be processed at once:
\begin{verbatim}
       ahab% foreach file (*.F)
       foreach? redo < $file > $file.red
       foreach? findent < $file.red > $file.fin
       foreach? relabel < $file.fin > $file.rel
       foreach? echo $file done
       foreach? end
\end{verbatim}
You can then rename \code{\$file.rel} to \code{\$file.F} and get rid of
the temporary files {\sl after} checking to make sure that all the
files still look sensible.

\subsection{\code{unenddo}}
\code{unenddo} will turn all \code{do-enddo} loops into
\code{do-continue} loops to comply with the Fortran 77 standard.  It is
used as follows:
\begin{verbatim}
       unenddo < file.F > file.new
\end{verbatim}
\code{unenddo} replaces \code{enddo} statements with labelled
\code{continue} statements.  It starts numbering these statements at
2000 assuming that existing labels use only three digits.  If
desired, \code{unenddo} can be told to start labelling with a
different number by modifying the \code{\$label\_no\_start}
variable.

\subsection{\code{ifspace}}
When I am feeling particularly contentious I also run \code{ifspace} on
the code.  It will convert
\begin{verbatim}
         if(i.eq.0.or.j.eq.0)   then
\end{verbatim}
to
\begin{verbatim}
         if (i .eq. 0 .or. j .eq. 0) then
\end{verbatim}
Use as follows:
\begin{verbatim}
       ifspace < file.F > file.new
\end{verbatim}

\subsection{\code{sfmakedepend}}
\label{sfm}
The other \code{Perl} script I use with Fortran modifies the
\code{Makefile} to include dependency information, much like the X11
program \code{makedepend}.  I originally wrote \code{fmakedepend} which
was used with traditional Fortran include statements.
I later wrote a variant of it for use with the C preprocessor,
called \code{sfmakedepend}.  The latest version of
\code{sfmakedepend} does the job of both programs and also searches for
the dependencies introduced by Fortran 90 modules.  It is used by the
\code{Makefiles} described in \S\ref{Imake}.

It recursively searches for Fortran
style includes, for instance if \code{file.f} has the statement:
\begin{verbatim}
           include 'commons.h'
\end{verbatim}
the line
\begin{verbatim}
        file.o: commons.h
\end{verbatim}
will be added to the bottom of the \code{Makefile}.  This tells
\code{make} that \code{file.o} depends on \code{commons.h} as well
as \code{file.f}, and to recompile \code{file.f} whenever
\code{commons.h} is modified.
It likewise searches source files for C style includes such as
\begin{verbatim}
       #include "commons.h"
\end{verbatim}
and adds the corresponding dependencies to the \code{Makefile}.
It has several options, including \code{-s}, required for Fortran
compilers which will not invoke the C preprocessor for you.  In this
case the above dependency line would become
\begin{verbatim}
        file.o: commons.h
        file.f: commons.h
\end{verbatim}
letting \code{make} know that the C preprocessor must be rerun on
\code{file.F} whenever \code{commons.h} is updated.

When using the C preprocessor, you can ask it to search directories
other than the current directory.  Likewise, \code{sfmakedepend} can be
instructed to search other directories with \code{-I dir} options.
Note that it is legal to have more than one \code{-I dir} option as in:
\begin{verbatim}
      sfmakedepend -I /usr/local/include -I /home/me/include *.F
\end{verbatim}

Fortran 90 introduces some interesting dependencies.  Two compilers I
have access to (NAG \code{f90} and IBM \code{xlf}) produce a private
\code{my\_module.mod} file if you define \code{module}
\code{My\_Module} in file \code{mod.f90}.  This file is used by the
compiler when you use the module as a consistency check (type-safe
programming).  If \code{foo.f90} uses that module, you will need the
following dependency information:
\begin{verbatim}
      foo.o: my_module.mod
      my_module.mod: mod.o
\end{verbatim}
This says that before compiling \code{foo.f90} we need to have the file
\code{my\_module.mod}.  This file in turn depends on \code{mod.o}, so
that \code{mod.f90} must be compiled before \code{foo.f90}.
The sgi is similar except that it uses the file \code{MY\_MODULE.kmo}
to store the private module information.  Use \code{sfmakedepend -g}
on the SGI.

Rather than creating extra module files,
the Cray and Parasoft compilers store the module information in the
object file and then files which use the modules need to be compiled
with extra flags pointing to the module object files.  For instance, if
\code{foo.f90} uses \code{My\_Module} which was defined in
\code{mod.f90}, then you will need to compile \code{mod.f90} first and
provide the Cray compiler with the extra option \code{-p mod.o} when
compiling \code{foo.f90}.  When using the Cray, use \code{sfmakedepend
-c} to get the dependency information:
\begin{verbatim}
       foo.o: mod.o
               $(CFT) $(FFLAGS) -c -p mod.o foo.f90
\end{verbatim}
\code{\$(CFT)} and \code{\$(FFLAGS)} are assumed to be previously
defined as the name of the compiler and the compiler options,
respectively.

{\bf Note:} These f90 module dependencies can confuse some versions of
\code{make}, especially of the System V variety.  We use gnu
\code{make} because it can follow these chained dependencies and do the
right thing.

\code{sfmakedepend} assumes that all the files using and defining
modules are in the same directory and are all in the list of
files to be searched.  It seems that the industry has not
settled on a practical way to deal with a separate modules
directory, anyway.

I sometimes include non-existent files as a compile time
consistency check:
\begin{verbatim}
        #ifndef PLOTS
        #include "must_define_PLOTS"       /* bogus include */
        #endif
\end{verbatim}
This program warns about include files it can't find, but
not if there is a ``bogus'' on the same line.

See the comments at the top of \code{sfmakedepend} for up-to-date
information on the options.  I may someday get inspired to use a
newer version of the \code{getopt} routine and rename the options
to have names like \code{-SGI} and \code{-Cray}.
